% vim:set spell tw=79:

\documentclass[article]{uibk}
\title{Exploitation Techniques and Mitigations}
\author{Alex Hirsch \and Patrick Ober}
\date{2016-01-31}

\begin{document}

\maketitle

% \section*{Abstract}
% \label{sec:abstract}

% \tableofcontents

% \newpage

\section*{Acknowledgement}

A university course at Rensselaer Polytechnic
Institut\footnote{\url{http://rpi.edu/}} held in Spring 2015 focused on
\textit{Modern Binary Exploitation}. They made their course material available
on GitHub \cite{RPISEC} under the Creative Commons Attribution-NonCommercial
4.0 International
license\footnote{\url{https://creativecommons.org/licenses/by-nc/4.0/legalcode}}.
We reused a lot of their material in this project.

We highly recommend checking them out and having a look at their material for
further details.

\section{Introduction}

Exploiting binaries was comparatively easy in the early days of computing.
Usually there were no special mitigation techniques in place trying to prevent
even the most simplest exploits. This is the point in time where we will start
of. First we talk about two very simple exploits, namely the Format String
Exploit and the Buffer Overflow in combination with Shell Code. Note that there
is a huge collection of exploitation techniques known to the public and we will
thereby only look at a very small fraction of them.

But before we can introduce these two exploits, some background knowledge is
required. This will be handled by the next section, which provides a short
overview of the relevant components in our target architecture, the x86
platform.

After that both techniques are introduced to the reader, followed by the first
mitigation technique, Data Execution Prevention (DEP). From there on we will
keep on using the buffer overflow technique with some adaptations to circumvent
DEP. At this point Return Oriented Programming (ROP) is introduced.

This directly leads to Address Space Layout Randomization (ASLR) the next
mitigation mechanism we will discuss. Again the buffer overflow technique can
be adapted to break ASLR through the use of additional information.

Since neither DEP nor ASLR provide significant protection against even this
simple technique, an additional mitigation is put into place in the form of
Stack Cookies.

Examples will be provided along the way to support the reader and provide some
additional explanation.

Control Flow Integrity (CFI), Heap Corruption and polymorphic code will follow
in a more compressed manner to communicate the main idea behind each of them.

Finally we will conclude with a word about other architectures (x86\_64 and
ARM) and a lookout that even languages considered secure have their own set of
exploitation techniques an attacker could leverage.

\section{Platform x86}

This section will teach necessary background knowledge about the target
platform to fully conceive the following techniques. But first let us elaborate
why x86 has been chosen in the first place.

At the time these techniques (and the related mitigations) were established,
x86 was the most common platform. Since most exploits easily translate over
from x86 to other architectures, especially x86\_64 which very common nowadays.
Also, most material found on the internet regarding this and related topics
cover x86.

More detailed explanations can be found on
Wikipedia\footnote{\url{https://en.wikipedia.org/wiki/X86}} or the Intel
Manual\footnote{\url{https://www-ssl.intel.com/content/www/us/en/processors/architectures-software-developer-manuals.html}}.

\subsection{CPU and registers}

\subsection{Memory}

\newpage

\bibliography{references}

\end{document}
